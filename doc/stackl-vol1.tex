\documentclass[11pt,a4paper,oneside]{book}

% ---- Page Layout ----
\usepackage[margin=1in]{geometry}

\usepackage{booktabs}

% Better PDF links
\usepackage[hidelinks]{hyperref}

% ---- Fonts ----
\usepackage{palatino}

% Fix tables drifting to bottom
\usepackage{float}
\restylefloat{table}


\makeatletter
\renewcommand{\maketitle}{%
  \vspace*{2cm} % Add some space from the top
  \begin{center}
    \begingroup
      \hrule height 2pt % Optional: Thick line above the title
      \vspace{0.5\baselineskip}
      {\Huge \bfseries \@title \par}% Title is large and bold
      \vspace{0.5\baselineskip}
      \hrule height 1pt % Optional: Thin line below the title
    \endgroup
    \par
  \end{center}
  \vfill % Push everything below this point to the bottom of the page
  % --- Author and Date at Bottom ---
  \begin{center}
    \begingroup
      \footnotesize
      \hrule height 0.5pt % Thin separator line
      \vspace{0.5em}
      \@author \quad $\cdot$ \quad \@date % Print Author and Date separated by a dot
      \vspace{0.5em}
    \endgroup
  \end{center}
}
\makeatother



\title{
	STACKL Architecture\\
	Programmer's Manual\\
	\large Volume 1: Basic Architecture
}

\author{Jonathan Thomason}

\begin{document}

\frontmatter
\thispagestyle{empty}
\maketitle

\setcounter{page}{0}
\tableofcontents

\chapter{Preface}

\section{About This Book}
This book is part of a multivolume work entitled the \textit{STACKL Architecture
Programmer's Manual}. This table lists each volume in the series.

\begin{table}[H]
\centering
	\begin{tabular}{l}
		STACKL Architecture Programmer's Manual Volume 1: Basic Architecture\\
		STACKL Architecture Programmer's Manual Volume 2: Instruction Set Reference\\
		STACKL Architecture Programmer's Manual Volume 3: System Programming Guide\\
	\end{tabular}
\end{table}

\section{Audience}
This volume is intended for system architects, hardware implementers,
interpreter developers, and others requiring a precise understanding of
the STACKL architectural model.

\section{Scope of This Volume}
This volume describes the architectural state visible to software running on
STACKL, including the organization of its stacks, registers, program counter,
and condition flags.

It defines the execution model, memory address space, and interrupt vector
mechanism, establishing the foundation for software development on the
STACKL virtual machine.

Details of instruction encoding and semantics are covered in *Volume 2:
Instruction Set Reference*, while system-level programming conventions and
runtime environment are described in *Volume 3: System Programming Guide*.

\mainmatter
% your content goes here

\chapter{Registers}
Overview of registers and their usage.

\chapter{Program Counter}
Explanation of how the program counter operates.

\chapter{Condition Flags}
Description and use cases for condition flags.

\chapter{Execution Model}
Detailed explanation of the execution process.

\chapter{Memory Model}

\chapter{Procedure Calls}

\chapter{Interrupts}

\chapter{Machine Checks}

\chapter{Control Transfers}

\chapter{Input/Output}

\chapter{Overview of the CPU Instruction Set}

\chapter{Overview of the FPU Instruction Set}

\chapter{Instruction Bit Encodings}

\end{document}
